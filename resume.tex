\documentclass[11pt,line,center]{res}
\usepackage{microtype}
\usepackage{url}
\usepackage{tabulary}

\renewcommand{\sectionfont}{\scshape}

\begin{document}
\name{Darrin J. Dykes}
\address{ddykes2@gmail.com | 971-238-4828 | github.com/rsq2}
\begin{resume}

\bigskip

\section{\sectionfontt\normalsize Relevant Work Experience}
    {\bfseries\footnotesize  Help Desk & Content Manager}, Oregon First Real Estate \hill \textit{Portland, OR Feb. 2015 - Current}\\
        \vspace{-5mm}
        \begin{itemize}
            \setlength\itemsep{1em}
            \item Scripted an automation using Ruby, Selenium, and the Quickbase API to cross reference company databases with the MLS,
                  retroactively auditing previous years of manual entry for the first time at zero cost.\\
            \item Gathered specifications and developed style guidelines for customer facing profile pages. Implemented new CSS classes
                  and Javascript within the company WordPress to match new guidelines.\\
        \end{itemize}

    {\bfseries\footnotesize  Help Desk & Content Manager}, Oregon First Real Estate \hill \textit{Portland, OR Feb. 2015 - Current}\\
        \vspace{-5mm}
        \begin{itemize}
            \setlength\itemsep{1em}
            \item Scripted automations using Python and Selenium that aggregated public voting and legislative district data from state
                  websites, completing months of backlogged data entry in hours. \\
        \end{itemize}


\section{\sectionfontt\normalsize Projects}
{\bfseries\footnotesize Raytracing Engine}\hill \textit {Python}\\
    A basic raytracing engine in Python 2.7. This project features a recursive algorithm, extensive polymorphism
    / operator overloading, and clearly defined object classes. My goal for this project was to create a lightweight,
    extensible structure such that new geometry and textures can easily be added later. A fun note: when I
    started this, I had never worked with linear algebra.
{\bfseries\footnotesize  Google Maps Scraper\hill \textit {Ruby, Capybara, Selenium}\\
    A toy automation that will look up a .CSV’s address contents in Google Maps. Robust tests and error
    handling using Rspec to account to lookup failure and approximated addresses. This is a demo project I
    used to apply to Hack Oregon 2015, a volunteer program to expose and analyze public datastreams.
{\bfseries\footnotesize  Real Estate Transaction Notifier}\hill \textit {Python}\\
    A simple passthrough mailing system using the Gmail API that scans e-mails, extracts tagged variables,
    and reconstructs sanitized, branded e-mails for distribution. This was an independent project done for my
    current employer. Eliminated a cumulative 20 hrs/wk of labor (\$\15,000/yr in savings).

\section{\sectionfont\normalsize Technologies}
    \begin{tabulary}{RJ}
        {\bfseries Prog.Languages} & Proficient with Python, Ruby, Javascript | Familiar with Java, Haskell
        \\
        {\bfseries Web} & Familiar with Rails & MEAN stacks | Proficient HTML, CSS, jQuery skills | Experience developing with RESTful APIs
        \\
        {\bfseries Automation / Testing} & Paid experience coding automations using Selenium Webdriver. | Familiar with py.test with Python unittest. Rspec, and Cucumber Ruby testing frameworks.
        \\
        {\bfseries DB / Other} & Familiar with mySQL, noSQL (MongoDB) | Experience using \LaTeX\ for general purpose typesetting | Dedicated Linux user
\end{tabulary}
\end{resume}
\end{document}
